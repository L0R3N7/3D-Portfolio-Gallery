\subsection{Angular Services [M]}
\label{txt:sec:Services}

Um eine Schnittstelle zwischen dem Backend und dem Frontend herzustellen, werden \emph{Angular Services} verwendet.

\subsubsection{Allgemein}
In Angular entspricht ein Service einer TypeScript-Klasse, die verwendet wird, um Logik auszulagern, die von der gesamten Applikation verwendet werden kann. Services werden meist dann erstellt, wenn eine einfache Logik oft von verschiedenen Komponenten benötigt wird. Um die globale Verwendung zu realisieren, wird das Konzept der \emph{Dependency Injection} verwendet. \cite{AngularBuch} \cite{AngularArchitectureService}

\subsubsection{Dependency Injection}
\label{DPI}
In Angular ist es möglich, einen Service in eine beliebige Komponente zu injizieren. Das bedeutet, ein Service wird anhand von \emph{@Injectable()} definiert und kann dadurch von anderen Komponenten verwendet werden \cite{AngularBuch}:

\begin{lstlisting}[caption={Eine Klasse Injectable machen},  language=TypeScript,label=lst:impl:injectable]   
    
 @Injectable({
  providedIn: 'root'
 })
  export class GalleryService{
    ..
  }
   
\end{lstlisting}

Die Codezeile “providedIn: 'root'” sorgt dafür, dass sich Services entweder nur in spezifische Komponenten injizieren lassen oder sich wie hier auf \emph{root-level}, also überall, injizieren lassen. \cite{AngularBuch}

Ein Service wird zur Verwendung im Konstruktor einer Komponente initialisiert \emph{(Constructor Injection)}: 

\begin{lstlisting}[caption={Constructor Injection},  language=TypeScript,label=lst:impl:concstructorinjection]   
    constructor(private gs: GalleryService) { }
\end{lstlisting}
Anschließend können die Methoden und Daten eines Service ganz normal verwendet werden.

\subsubsection{HTTP Module}
\label{httpService}
\label{sec:HTTPModule}
Um mit dem Backend über das HTTP-Protokoll kommunizieren zu können, wird eine HTTP-API (siehe \ref{txt:glos:API}) namens \emph{HttpClient} verwendet. Zunächst muss das \emph{HttpClientModule} im \emph{ngModule} importiert werden, um den \emph{HttpClient}  als Dependency zu injizieren. Injiziert wird er über den Konstruktor in einem Service. Um eine Transaktion am Frontend durchzuführen, werden Observables verwendet. Dies ermöglicht den einzelnen Komponenten, die einen Service mit HTTP-Abfragen injiziert haben, diese Abfragen mittels einem \emph{Subscribe} zu nutzen. Ein \emph{Subscribe} ist ein Operator von \emph{RxJS} (siehe \ref{RxJS}). Eine HTTP-Abfrage mit dem \emph{HttpClient} wird wie folgt aufgerufen \cite{AngularBuch} \cite{AngularHTTPClient}:

\begin{lstlisting}[caption={HttpClient Abfragen},  language=TypeScript,label=lst:impl:httpclientrequests]   
    URL = "http://localhost:8080/api/"

    getAllRooms(): Observable<Room[]>{
        return this.httpClient.get<Room[]>(`${this.URL}rooms/allRoomPositions`);
      }
    
\end{lstlisting}

Wie anhand des oberen Code-Beispiels erkennbar ist, wird hier das von uns erstellte Room-Interface benutzt.

\subsubsection{Interface [M]}
\label{interface}
\emph{Interfaces} werden verwendet, um ein Objekt typsicher zu strukturieren. Dabei wird, um Daten des Servers korrekt erhalten zu können, eine bestimmte Entität der Datenbank verglichen und durch die richtige Benennung und der korrekte Datentyp im Frontend abgebildet. Üblicherweise wird ein solches Datenobjekt exportiert, um es in der ganzen Anwendung zu verwenden. Folgendes Beispiel demonstriert dieses Konzept anhand des Exhibit-Interfaces \cite{AngularBuch}: 

\begin{lstlisting}[caption={Das Datenmodell eines Ausstelungsstückes},  language=TypeScript,label=lst:impl:httpclientrequests]   
  
export class Exhibit{
    id : number;
    url : string;
    data_type : string;
    title : string;
    description : string;
    alignment: string | undefined
    position: Position | undefined
    scale: number | undefined
  
    constructor(id: number, model_url: string, data_type: string, title: string, desc: string, alignment: string | undefined, position: Position | undefined, scale: number | undefined) {
      this.id = id;
      this.url = model_url;
      this.data_type = data_type;
      this.title = title;
      this.description = desc;
      this.alignment = alignment
      this.position = position
      this.scale = scale
    }
}
\end{lstlisting}


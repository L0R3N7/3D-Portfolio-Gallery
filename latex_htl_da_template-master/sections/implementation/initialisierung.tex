\section{Initialisierung der Landingpage [L]}
\setauthor{Litzlbauer Lorenz}
Der erste Entwicklungsschritt für die vollständige Single-Page-Application beginnt mit der Initialisierung der Landingpage. Die Landingpage ist die Startseite der Webseite. Es ist das Erste, was der*die neue Nutzer*in sieht. Daher muss die Webseite alle benötigten Informationen über die Funktionen der Applikation verständlich erkennbar machen.

\subsection{Aufsetzen der Landingpage [L]}
\setauthor{Litzlbauer Lorenz}
Die Entwicklung startete damit, dass die benötigten Technologien Angular, AngularThree, AngularMaterials und Bootstrap heruntergeladen werden mussten.

Dafür wurde der \emph{npm} Node-Package-Manager verwendet.


\subsubsection{Vorbereitung}
Erstmal musste NodeJs installiert werden. Dafür wird aber zuvor noch NodeJs benötigt. NodeJs ist eine JavaScript Laufzeitumgebung. NodeJs kann von der eigenen Webseite nodejs.org mit dem Installer für alle Betriebssysteme installiert werden. Im Projekt wurde sich für eine LTS - Version(long term support) entschieden, weil diese am längsten von den Entwicklern am längsten unterstützt werden und dadurch beständiger sind. Während dem Installationsprozess von NodeJs kann durch eine Auswahl auch NPM installiert werden. 

%%\paragraph{npm - Node Package Manager}
%%Der Node Package Manager ist ein Softwareverwaltungstool zum Downloaden, Aktualisieren, Veröffentlichen und Verwalten von OpenSource-Programmen in der NodeJs-Umgebung. \cite{whatNpm} \cite{AboutNpm}

\subsubsection{Angular installieren}\label{sec:AngularCLI}
Angular hat ein eigenes Tool, die Angular CLI (Command Line Interface), um Projekte zu erstellen, zu bearbeiten, Komponenten, Services und vordefinierte Codemodule hinzuzufügen und das Projekt zu bauen.

\begin{lstlisting}[caption={{Terminalm - Angular aufsetzen, Installation der CLI, Configuration eines neuen Projektes, Starten des Projektes}},language=bash]
npm install -g @angular/cli 
ng new Gallery
? Would you like to add Angular routing? Yes
? Which stylesheet format would you like to use? SCSS   [ https://sass-lang.com/documentation/syntax#scss ]
cd Gallery
ng serve -o
\end{lstlisting}

In der ersten Codezeile wird das Angular-CLI-Tool global von NPM installiert.
Danach wird mit dem Befehl ng new mit dem CLI-Tool ein neues Angular Projekt erstellt. Danach wird es mehrere Konfigurationsauswahlmöglichkeiten geben. Für dieses Projekt wurden Routing aktiviert und  als Stylesheet-Formatierung Scss ausgewählt. Danach wurde in das (Projekt-) Verzeichnis Gallery gewechselt und dort mit dem Befehl serve der ein Webpack-Server gestartet, welcher den vorgenerierten Code von dem neu erstellen Angular-Projekt zeigt (siehe in Abbildung \ref{fig:impl:angular-starting-page}). 

\begin{figure}
    \centering
    \includegraphics[scale=0.25]{pics/AngularStartingPage.png}
    \caption{Angular: Automatisch generierte Start-Webseite}
    \label{fig:impl:angular-starting-page}
\end{figure}

\paragraph{Globale oder Lokale Module}
Bei einer lokalen Installation werden die installierten Module in einem node-module Computerordner lokal im Projekt abgespeichert, während alle globalen Installationen in einem einzigen Computerordner, abhängig vom Computersystem gespeichert werden.
Generell kann man bei NPM-Module zwischen lokalen und globalen Installationen unterscheiden. In der Regel ist eine lokale Installation besser, denn referenzieren mehrere Projekte auf ein globales Modul, kann es dazu kommen, dass bei einer Aktualisierung des globalen Modules verschiedene Projekte, sei es wegen veralteten Funktionen oder neuer Logik, darauf anders reagieren und es zu Problemen bei diesen Projekten kommt, da nichts im Projekt darauf referenzieren muss. 
CLI-Module können aber auch lokal installiert werden und mit dem Befehl npx nur im Projektordner ausgeführt  werden. 
\cite{npmlocalorglobal}

\subsubsection{Installation der UI-Framework}

Nach der Installation von Angular wurden die UI-Frameworks installiert. Angular Material konnte mithilfe des Befehls  \ref{lst:impl:installationAngularMaterials} mittels der Angular CLI in das Angularprojekt eingebunden werden.

\begin{lstlisting}[caption={{Terminal - Angular Material Installation}},language=bash,label=lst:impl:installationAngularMaterials]
    ng add @angular/material
\end{lstlisting}

Bootstrap konnte mithilfe des Befehls \ref{lst:impl:installationBootstrap} und von NPM installiert werden. Bootstrap wurde zwar dem Projekt hinzugefügt, Angular weiß aber davon noch nichts. Deswegen musste in der Anuglar-Konfigurationsdatei \emph{angular.json} die Bootstrap Scss-Liberay eingebunden werden. Das wird in diesem Code \ref{lst:impl:BootstraptConfig} veranschaulicht. 

\begin{lstlisting}[caption={{Terminal - Bootstrap Installation}},language=bash,label=lst:impl:installationBootstrap]
    npm i bootstrap
\end{lstlisting}

\begin{lstlisting}[caption={{angular.json - Bootstrap Angular Verknüpfung}},label=lst:impl:BootstraptConfig]
    {
        ...,
        "projects": {
            "Gallery": {
                ...,
                "architect": {
                    "build": {
                        ...,
                        "options": {
                            ...,
                            "styles": [
                                ...,
                                "node_modules/bootstrap/scss/bootstrap.scss"
                            ]
                        }
                    }
                }
            }
        },
        ...
    }
    
\end{lstlisting}


\subsubsection{Installation von Three Js}
Nach der Installation der UI-Libraries wurde ThreeJs installiert. 

Mit dem ersten Befehl wird AngularJs eine JavaScript Liberay durch NPM installiert und durch den zweiten Befehl wurde ein Typelibary für ThreeJs installiert um mit Typescript ThreeJs bearbeiten zu können.
\begin{lstlisting}
    npm install three
    npm i @types/three
\end{lstlisting}

\subsection{Components [M]}
\label{components}
\setauthor{Fabian Maar}
Der nächste Schritt der Entwicklung ist das Erstellen der Komponenten und das Festlegen ihrer Struktur. Eine Komponente kann ebenfalls über die Angular CLI erstellt werden:

\begin{lstlisting}[caption={{Terminal - Component erstellen}},language=bash,label=lst:impl:addComponent]
    ng generate component home-page
\end{lstlisting}

\subsubsection{Allgemein}
Angular Komponenten sind die Bausteine für eine Angular-Anwendung. Mit ihnen lässt sich eine komplexe Benutzeroberfläche in mehrere unterschiedlich große Elemente unterteilen. Sie lassen sich mit einem eigenen HTML-Selektor (z.B. <my-component>) ansprechen und in die Benutzeroberfläche einbinden. Components besitzen viele Features, die den Entwicklungsprozess, Wartung des Codes und Fehlersuche erleichtern können:

\begin{itemize}
    \item Trennung von Logik und Design
    \item Skalierbarkeit der Applikation
    \item Data-Binding
    \item Hierarchie
    \item Lesbarkeit/Übersichtlichkeit
\end{itemize}

\subsubsection{Aufbau}
Um die Logik strikt von der Benutzeroberfläche zu trennen, werden die Dateien und der Code innerhalb einer Komponente strukturiert und separiert. Eine Komponente besteht somit aus: 

\begin{itemize}
    \item einem HTML-Template, dass die Benutzeroberfläche darstellt
    \item einer TypeScript-Klasse, die die Logik und Funktionalitäten der Komponente beinhaltet
    \item ein Stylesheet, dass das Aussehen der Benutzeroberfläche beeinflusst
    \item eine TypeScript-Testklasse, um individuell Komponenten zu testen 
\end{itemize}

\cite{AngularComponentOverview}

\subsubsection{Skalierung}
Die Skalierung beschreibt, wie gut eine Applikation mit Erweiterungen und ihrem Wachstum umgeht. Dabei wird geschaut, wie leicht sich zusätzliche Änderungen und Features implementieren lassen oder wie sich die Performanz der Anwendung bei zunehmender Größe verhält. Angular hilft beim Skalieren durch einige Features:

TypeScript hilft durch Code-Syntax, wie unter anderem Optionals, auch bei großen Applikationen schnell Fehler zu erkennen. 

Durch die Angular CLI kann durch wenige Zeilen Befehle, zum einen ein Grundgerüst für die Entwicklung erstellt werden. Andererseits können Components, Services und andere Angular Funktionen in jedem Entwicklungsstadium problemlos hinzugefügt werden, ohne dass der bestehende Code beeinflusst wird. siehe Kapitel \ref{sec:AngularCLI}

Angular stellt zudem Libraries wie Angular Materials oder das Component Development Kit (cdk) zur Verfügung. Dadurch können problemlos vorgefertigte Angular Komponenten und Designs zu einer Applikation hinzugefügt werden.

\subsubsection{Hierarchie}
Die Hierarchie von Angular Komponenten lässt sich wie folgt in einer Baumstruktur abbilden:

\begin{figure}
    \centering
    \includegraphics[scale=0.6]{pics/hierarchy.PNG}
    \caption{Component-Hierarchy \cite{AngularBuch}}
    \label{fig:tech:front:component-hierarchy}
  \end{figure}
Die Hauptkomponente ist in jedem Falle der Root-Component. Von ihr aus lassen sich weitere Child-Components ineinander verschachteln. Diese Verschachtelung ist der Grund, warum die Anwendung in viele einzelne Teile ausgelagert werden kann. Außerdem können Komponenten durch das Data-Binding untereinander Daten austauschen.

\subsubsection{Data-Binding }

Data-Binding ist eine Methode, Daten zwischen der Logik und der Benutzeroberfläche auszutauschen. Dabei bleibt die Website ohne Refresh immer aktuell. Das Binden von Daten kann direkt am DOM erfolgen, wofür es eine eigene Code-Syntax gibt. Beim Data-Binding wird zwischen  verschiedenen Kategorien unterschieden \cite{AngularBuch} \cite{BindingSyntax}:  
    

\paragraph{Interpolation}
Bei der Interpolation werden Daten innerhalb einer Komponente von der Datenquelle in das HTML-Template eingefügt. Die Anzeige wird auch automatisch aktualisiert, wenn sich Daten im Hintergrund ändern.

\begin{lstlisting}[caption={{Beispiel für Interpolation in der 3D-Gallery}},language=HTML,label=lst:impl:interpolation]
    <div>
    <p class="text-center py-2">{{exhibit.title}}</p>
    </div>
\end{lstlisting}

\paragraph{Property-Bindings}
Beim Property-Binding werden Daten direkt an ein DOM-Element übermittelt und ausgewertet. Somit werden die Attribute, Aussehen und Funktionen der jeweiligen DOM-Elemente dynamisch beeinflusst und automatisch bei Datenänderung aktualisiert. \cite{AngularPropertyBinding}
\begin{lstlisting}[caption={{Beispiel für Property-Bindings in der 3D-Gallery}},language=HTML,label=lst:impl:property-binding]
    <img [src]="user?.icon_url ?? 'assets/image/profile-photo.jpg'">
\end{lstlisting}

\paragraph{Event-Bindings}
Um auf Eingaben und Interaktionen von Benutzer*innen zu reagieren, werden Event-Bindings verwendet. Dabei werden die Daten vom HTML-Template zur TypeScript-Klasse übertragen und können dort genutzt werden. Sie sind also das Gegenstück zu den Property-Bindings. \cite{AngularEventBinding}
\begin{lstlisting}[caption={{Beispiel für Event-Bindings in der 3D-Gallery}},language=HTML,label=lst:impl:event-binding]
    <button(click)="delete()">
      <mat-icon>close</mat-icon>
    </button>
\end{lstlisting}

\paragraph{Two-Way-Bindings}
Das To-Way-Binding benutzt beide Varianten, Property- und Event-Binding, um Daten auszutauschen. Hierbei ist es möglich, die Daten sowohl von der TypeScript-Klasse zum DOM-Element zu übertragen und umgekehrt. \cite{AngularTwoWayBinding}
\begin{lstlisting}[caption={{Beispiel für Two-Way-Bindings \cite{AngularTwoWayBinding}}},language=HTML,label=lst:impl:two-way-bindings]
    <app-sizer [(size)]="fontSizePx"></app-sizer>
\end{lstlisting}
In den folgenden Kapitel werden die Komponenten und ihre Funktionalitäten näher beschrieben. 

\subsection{Routing [M]}\label{sec:Routing}
Beim Routing werden bestimmte Components angezeigt, abhängig von der aktuellen URL. Dabei kann durch die Applikation durch navigiert werden, was durch den sogenannten Router realisiert wird. Routing ist das Konzept einer Single-Page-Applikation, wodurch die Seite niemals neu geladen werden muss und alle Daten beim Navigieren beibehalten werden können. Um das Routing überhaupt verwenden zu können, müssen die Pfade zu den zugehörigen Components initialisiert werden. Dies geschieht in einem ngModule, wo alle initialisierten Routen über das RouterModule importiert werden. Um die Funktionalitäten und Routen für alle Components verwenden zu können, muss dieses RouterModule ebenfalls exportiert werden. Der Code zeigt dies anhand der 3D-Gallerie-Anwendung:
\begin{lstlisting}[caption={Routing in der 3D-Gallery},language=TypeScript,label=lst:impl:routing]
const routes: Routes = [
  {path: '', component:HomePageComponent},
  {path: 'home', component:HomePageComponent},
  {path: 'log-signin', component:LogSigninPageComponent},
  {path: 'search', component:SearchPageComponent},
  {path: 'profile', component:ProfilePageComponent},
  {path: 'create', component:CreateExhibitionPageComponent},
  {path: 'signup', component:SignupPageComponent},
  {path: 'room/:id', component:RoomPageComponent},
  {path: '**', redirectTo: 'home'}
  ];
  @NgModule({
    declarations: [],
    imports: [ CommonModule, RouterModule.forRoot(routes) ],
    exports: [ RouterModule ]
  })

\end{lstlisting}
Hierbei steht der Pfad ‘**’ für alle ungültigen URLs wodurch der*die Benutzer*in auf die Landingpage geleitet wird.

\subsubsection{Routingparameter}
\label{Routingparameter}
Um Routen dynamisch zu deklarieren, werden sogenannte Routenparameter benötigt. Hierbei beginnt der dynamische Teil einer Route mit einem Doppelpunkt. 


\begin{lstlisting}[caption={Routingparamter in der 3D-Gallery},language=TypeScript,label=lst:impl:routingparameter]
    const routes: Routes = [
      {path: 'room/:id', component:RoomPageComponent},
      ];    
\end{lstlisting}

Um den aktuellen Zustand der URL auszulesen, muss der Router im Konstruktor injiziert werden. Anschließend kann der Parameter wie folgt ausgelesen und als Wert einer Variable zugewiesen werden:

\begin{itemize}
    \item Entweder über die Snapshot-Methode, die den momentanen Zustand der URL ausliest 
    \begin{lstlisting}[caption={Snapshot der URL abfragen},language=TypeScript,label=lst:impl:routingsnapshot]
        this.id = this.route.snapshot.paramMap.get('id');
    \end{lstlisting}
    \item oder wie es in der 3D-Gallerie Anwendung gelöst ist, über das Observable-Pattern auf den Router subscriben, um auf ständige Änderungen im Pfad zu reagieren:
    \begin{lstlisting}[caption={Die URL subscriben},language=TypeScript,label=lst:impl:urlsubscription]
        this.sub = this.route.params.subscribe(params => {
            this.id = +params['id'];
          })
    \end{lstlisting}
\end{itemize}
\cite{AngularBuch}

\section{Ideenfindung [M]}
\setauthor{Fabian Maar}
Die Ideenfindung ist der Start eines Projektes und bildet das Fundament, auf dem das Projekt entsteht. Die Idee für die Entstehung eines Projektes kann viele Motive haben, so ist es wichtig, als Projektteam diese zu erkennen und als Motivation zu nehmen. Während oder nach dem Prozess der Ideenfindung wird schon bald klar, welche Projektwürdigkeit das Projekt besitzt. Somit kann schon frühzeitig entschieden werden, ob es sinnvoll erscheint, das Projekt zu realisieren oder ob nochmals eine neue Ideenfindung vonnöten ist. Auch bei dieser Diplomarbeit wurde der Prozess der Ideenfindung mehrmals wiederholt, wodurch die Wichtigkeit und die Methodik der Findung von Ideen besonders bemerkbar wurde. Die Ideenfindung erfolgt meist mithilfe der Anwendung einer passenden Kreativitätstechnik.

Kreativitätstechniken sind vielseitig einzusetzen, werden aber häufig am Anfang eines Projektes angewandt. Sie sind besonders hilfreich, wenn ein rationales Problemlösen nicht zielführend erscheint. Sie helfen, Ziele, Lösungen und Risiken zu finden, evaluieren und festzulegen. Für die Diplomarbeit bot sich besonders die Kreativitätstechnik Brainstorming an, da durch sie viele verschiedene Ideen in kürzester Zeit gesammelt werden können.

Beim Brainstormen findet jede*r Teilnehmer*in zu einem bestimmten Thema so viele Ideen wie möglich. Das Brainstormen erfolgt in 2 Phasen. Zuerst werden alle Ideen niedergeschrieben und anschließend die Beste herausgefiltert. Aufgrund der Expertise der Betreuungslehrerinnen im Bereich 3D-Modellierung und aus eigenem Interesse, konnte sich relativ schnell auf das Thema \emph{eine Software mit 3D-Integration} geeinigt werden. Schlussendlich konnte sich für eine Idee entschieden werden, nachdem das Brainstormen nach folgenden Kriterien erfolgt hat:
\begin{itemize}
    \item Quantität vor Qualität
    \item Freies Assoziieren und Fantasieren sind erwünscht
    \item Keine Kritik, Korrektur oder Meinungsäußerung
    \item Möglichst viele Ideen
    \item Inspirieren lassen von anderen
\end{itemize}
\cite{Ideenfindung}

\section{User-Stories [L]}
\label{ch:user-stories}
\setauthor{Litzlbauer Lorenz}
User Stories stellen Anforderungen an die Software. Sie werden user- und aufgabenorientiert formuliert und erzählen von der ersten Perspektive einer im Projekt involvierten Person, meistens des Users. \cite{AgileVorgehensmodelle}
 
Hier ein konkretes Beispiel, wie eine User Story gestaltet werden könnte: 
\emph{Als Designer möchte ich einen Login, um meine Ausstellungen speichern und im Nachhinein immer öffnen zu können.}

Die User Stories wirken als Kommunikationsmittel zwischen Kunde und Entwickler und bieten eine messbare Metrik zum Testen des Projektfortschritts. \cite{AgileVorgehensmodelle}

User Stories kommen oft in der agilen Softwareentwicklung vor. Dort besprechen der Kunde und der Product Owner eine Zielbestimmung. Anhand dieser Bestimmungen werden Funktionen bestimmt, die das Projekt erfüllen muss und diese mithilfe von User Stories formuliert. \cite{AgileVorgehensmodelle}

Auch für dieses Projekt wurden User Stories formuliert.
In weiterer Folge wird in dem Kapitel \ref{chapter:implementation} die Umsetzung der User Stories beleuchtet.





%%Im Projekt wurden mehrere Prototypenarten im Design- und Entwicklungsprozess verwendet. Damit  schnell ausgetestet, von den unterstützenden Professoren schnell Feedback eingeholt und aufgrund diesen Feedback Fehler schnell behoben werden konnten. 
%%
%%Anfangs wurde so die Machbarkeit des Projekts mit einer Umsetzbarkeitsanalyse, bei der der Prototyp eine große Rolle spielte, getestet. Im Prototyp (siehe Seite \pageref{ch:Technologien:AngularThree} Angular Three) wurde überprüft, ob die Technologien auf die das Projekt aufbaut auch die Anforderungen (siehe Seite \pageref{ch:umsetzung:projekt-user-stories} Projekt User Stories) erfüllen konnten. Dadurch konnten die Erkenntnisse gemacht werden, dass Angular Three nicht geeignete für das Projekt war und eine anderen 3D-Web-Api ThreeJs benutzt werden. Da dieses Resultat schon früh in dem Entwicklungsprozess gemacht wurde, kam mit der Änderung der 3D-Web-Apis im nur wenig Programmieraufwand zusätzlicher Aufwand.

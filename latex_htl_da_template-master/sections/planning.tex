
\section{Ideenfindung [M]}
\setauthor{Fabian Maar}

Die Idee des Projektes entstand durch das bestehende Interesse an der 3D-Modellierung. Besonders der Trend \emph{virtuelle Welten}begeisterte das gesamte Diplomarbeitsteam. Auch war klar, dass das Projekt ebenfalls in der Medientechnik zum Einsatz kommen sollte. Die Grundidee war eine Anwendung zu schaffen, die unsere Medientechnik-Projekte innovativ präsentiert.

Um dies genauer zu spezifizieren, erfolgte die Ideenfindung mithilfe der Anwendung einer passenden Kreativitätstechnik.Kreativitätstechnik sind besonders hilfreich, wenn ein rationales Problemlösen nicht zielführend erscheint. Sie helfen Ziele, Lösungen und Risiken zu finden, evaluieren und festzulegen. Für die Diplomarbeit bot sich besonders die Kreativitätstechnik Brainstorming an, da durch sie viele verschiedene Ideen in kürzester Zeit gesammelt werden können. \cite{Ideenfindung}

Beim Brainstormen findet jede*r Teilnehmer*in zu einem bestimmten Thema so viele Ideen wie möglich. Das Brainstormen erfolgt in 2 Phasen. Zuerst werden alle Ideen niedergeschrieben und anschließend die Beste herausgefiltert.  

Beim Filterprozess waren uns folgende Kriterien wichtig:
\begin{itemize}
    \item Quantität vor Qualität
    \item Freies Assoziieren ist erwünscht
    \item Keine Kritik, Korrektur oder Meinungsäußerung
    \item Möglichst viele Ideen
    \item Inspirieren lassen von anderen \cite{Ideenfindung}
\end{itemize}


\subsection{Namensfindung [L]}
Der Fokus bei der Namensfindung lag darauf, mittels des Namens der Diplomarbeit schon einen Eindruck auf die Funktionalitäten und den Umfang dieser zu geben.

Für die Namensfindung wurde die Kreativitätstechnik Brainstorming benutzt. Die Teammitglieder trafen sich und brachten Ideen ein.

Schlussendlich wurde sich für den Namen \emph{3D Portfolio Gallery} entschieden.

Dass wir im Projektnamen das Wort Gallery benutzen, hat folgende Gründe: 
\begin{compactitem}
    \item In einer Gallery werden Gegenstände oder Bilder oder künstlerische Installationen präsentiert.
    \item Eine Gallery ist nach Themen und oder Räumen aufgeteilt.
    \item Eine Gallery hat ein Orientierungssystem, um die Besucher*innen durch die Ausstellung zu leiten.
    \item die virtuell begehbaren Räume und die darin befindlichen präsentierten Objekte setzen die Ausstellungsstücke in eine örtliche Beziehung zueinander
\end{compactitem}



\section{User Stories [L]}
\label{ch:user-stories}
\setauthor{Litzlbauer Lorenz}
User Stories stellen konkrete Anforderungen an die Software. Sie werden user- und werden aus der Perspektive einer einer im Projekt involvierten Person, meistens des Users, beschrieben. \cite{AgileVorgehensmodelle}
 
Hier ein konkretes Beispiel, wie eine User Story gestaltet werden könnte: 
"\emph{Als Designer möchte ich einen Login, um meine Ausstellungen speichern und im Nachhinein immer öffnen zu können.}"

Die User Stories wirken als Kommunikationsmittel zwischen Kunde und Entwickler und bieten bieten auch eine Möglichkeit, den Projektfortschritt messbarer zu machen. \cite{AgileVorgehensmodelle}

User Stories kommen vor allem in der agilen Softwareentwicklung vor. Dort besprechen der Kunde und der Product Owner eine Zielbestimmung. Anhand dieser Bestimmungen werden Funktionen festgelegt, die das Projekt erfüllen muss und diese mithilfe von User Stories formuliert. \cite{AgileVorgehensmodelle}

In weiterer Folge wird in dem Kapitel \ref{chapter:implementation} die Umsetzung der User Stories beleuchtet.





%%Im Projekt wurden mehrere Prototypenarten im Design- und Entwicklungsprozess verwendet. Damit  schnell ausgetestet, von den unterstützenden Professoren schnell Feedback eingeholt und aufgrund diesen Feedback Fehler schnell behoben werden konnten. 
%%
%%Anfangs wurde so die Machbarkeit des Projekts mit einer Umsetzbarkeitsanalyse, bei der der Prototyp eine große Rolle spielte, getestet. Im Prototyp (siehe Seite \pageref{ch:Technologien:AngularThree} Angular Three) wurde überprüft, ob die Technologien auf die das Projekt aufbaut auch die Anforderungen (siehe Seite \pageref{ch:umsetzung:projekt-user-stories} Projekt User Stories) erfüllen konnten. Dadurch konnten die Erkenntnisse gemacht werden, dass Angular Three nicht geeignete für das Projekt war und eine anderen 3D-Web-Api ThreeJs benutzt werden. Da dieses Resultat schon früh in dem Entwicklungsprozess gemacht wurde, kam mit der Änderung der 3D-Web-Apis im nur wenig Programmieraufwand zusätzlicher Aufwand.

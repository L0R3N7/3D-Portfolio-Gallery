
\section{Ideenfindung [M]}
\setauthor{Fabian Maar}
Die Ideenfindung ist der Start eines Projektes und bildet das Fundament, auf dem das Projekt entsteht. Die Idee für die Entstehung eines Projektes kann viele Motive haben, so ist es wichtig, als Projektteam diese zu erkennen und als Motivation zu nehmen. Während oder nach dem Prozess der Ideenfindung wird schon bald klar, welche Projektwürdigkeit das Projekt besitzt. Somit kann schon frühzeitig entschieden werden, ob es sinnvoll erscheint, das Projekt zu realisieren oder ob nochmals eine neue Ideenfindung vonnöten ist. Auch bei dieser Diplomarbeit wurde der Prozess der Ideenfindung mehrmals wiederholt, wodurch die Wichtigkeit und die Methodik der Findung von Ideen besonders bemerkbar wurde. Die Ideenfindung erfolgt meist mithilfe der Anwendung einer passenden Kreativitätstechnik.

Kreativitätstechniken sind vielseitig einzusetzen, werden aber häufig am Anfang eines Projektes angewandt. Sie sind besonders hilfreich, wenn ein rationales Problemlösen nicht zielführend erscheint. Sie helfen, Ziele, Lösungen und Risiken zu finden, evaluieren und festzulegen. Für die Diplomarbeit bot sich besonders die Kreativitätstechnik Brainstorming an, da durch sie viele verschiedene Ideen in kürzester Zeit gesammelt werden können.

Beim Brainstormen findet jede*r Teilnehmer*in zu einem bestimmten Thema so viele Ideen wie möglich. Das Brainstormen erfolgt in 2 Phasen. Zuerst werden alle Ideen niedergeschrieben und anschließend die Beste herausgefiltert. Aufgrund der Expertise der Betreuungslehrerinnen im Bereich 3D-Modellierung und aus eigenem Interesse, konnte sich relativ schnell auf das Thema \emph{eine Software mit 3D-Integration} geeinigt werden. Schlussendlich konnte sich für eine Idee entschieden werden, nachdem das Brainstormen nach folgenden Kriterien erfolgt hat:
\begin{itemize}
    \item Quantität vor Qualität
    \item Freies Assoziieren und Fantasieren sind erwünscht
    \item Keine Kritik, Korrektur oder Meinungsäußerung
    \item Möglichst viele Ideen
    \item Inspirieren lassen von anderen
\end{itemize}
\cite{Ideenfindung}

\section{User-Stories}
\setauthor{Litzlbauer Lorenz}
User Stories stellen Anforderungen an die Software. Sie wird user- und aufgabenorientiert formuliert und erzählt von der ersten Perspektive einer im Projekt involvierten Person, meistens des Users.

Hier ein konkretes Beispiel, wie eine User Story gestaltet werden könnte.
\emph{Als Designer möchte ich einen Login, um meine Ausstellungen speichern und im Nachhinein immer öffnen zu können.}

Die User Stories wirken als Kommunikationsmittel zwischen Kunde und Entwickler und bieten eine messbare Metrik zum Testen des Projektfortschritts.

User Stories kommen oft in der agilen Softwareentwicklung vor. Dort besprechen der Kunde und der Product Owner eine Zielbestimmung. Anhand dieser Bestimmungen werden Funktionen bestimmt, die das Projekt erfüllen muss und diese mithilfe von User Stories formuliert.
\cite{AgileVorgehensmodelle}

Auch für dieses Projekt wurden User Stories formuliert. In diesem Kapiteln (Umsetzung) wird darüber geschrieben, wie diese erfüllt worden sind. 

\subsection{Projekt User Stories}
\setauthor{Team}
\label{ch:umsetzung:projekt-user-stories}
Die für das Projekt formulierten User Stories sind folgenden: 
\begin{compactenum}
    \item Als Designer möchte ich einen Login, um meine Ausstellungen speichern und im nachhinein immer öffnen zu können.
    Akzeptanzkriterien: 
    \begin{compactitem}
        \item Man kann sich neu registrieren
        \item Registrierung mittels Username und Passwort
        \item Das Passwort muss überprüft werden beim Erstellen und mind. 8 Zeichen enthalten
    \end{compactitem}  
    \item Als erstmalige*r Besucher*in der Webseite möchte ich die benötigten Informationen über die Funktionen der Applikation verständlich erkennen können. Auswahlkriterien: 
    \begin{compactitem}
        \item Auf der Landingpage befinden sich: 
        \begin{compactitem}
            \item Textstellen und Grafiken, die unser Projekt und die Funktionalitäten davon erklären
            \item Einen Call-to-Action(CTA)-Button, der den User*in dazu einlädt, seine eigene Ausstellung zu erstellen. Beim Betätigen wird der Benutzer, falls er schon eingeloggt ist, weitergeleitet zum Editor, sonst zur Login Seite (hier gibt es die Möglichkeit, sich auch einen Account zu erstellen)
        \end{compactitem}
    \end{compactitem} 
    \item Als Besucher*in möchte ich eine Suchseite haben, um Designer*innen und deren Ausstellungen finden zu können. Auswahlkriterien: 
    \begin{compactitem}
        \item CTA-Search Field
        \item Eine Unterseite, welche durch den CTA-Button aufgerufen wird und unterschiedliche Optionen zur Verfügung stellt und das Suchergebnis ansprechend darstellt
    \end{compactitem} 
    \item  Als Besucher*in der Webseite, will ich beim Suchen filtern können, um für sich die relevantesten Ergebnisse zu bekommen. Auswahlkriterien: 
    \begin{compactitem}
        \item filtern über …
        \begin{compactitem}
            \item Tags
            \item Favoriten
            \item Erstellungsdatum
        \end{compactitem}
    \end{compactitem} 
    \item Als Benutzer*in möchte ich meinen Ausstellungen Tags zuordnen können, damit diese leichter gefunden werden können. Auswahlkriterien: 
    \begin{compactitem}
        \item Tags werden beim erstellen der Ausstellung aus einen vordefinierten Tag-Pool ausgewählt.
    \end{compactitem} 
    \item Als User*in möchte ich mich auf verschiedene Arten in der Ausstellung bewegen können. Auswahlkriterien: 
    \begin{compactitem}
        \item Per Slideshow (über den Vorwärts/Rückwärts-Pfeil zu nächstem Ausstellungsstück springen)
        \item Per Touch / Click (Google Maps Street View, NavigationMesh in Three.js https://github.com/donmccurdy/three-pathfinding)
    \end{compactitem} 
    \item  Als User*in möchte ich auf das Ausstellungsstück klicken können, um weitere Informationen (z.B.: Titel, Künstler, Jahr, …) zu erhalten. Auswahlkriterien: 
    \begin{compactitem}
        \item größere Ansicht des Ausstellungsstücks wird vergrößert angezeigt
        \item Nähere Details zum Ausstellungsstück, wie Titel, Künstler, Jahr, usw.
    \end{compactitem} 
    \item  Als User*in will ich eine Profil-Unterseite haben, auf der ich userrelevante Informationen angezeigt bekomme. Auswahlkriterien: 
    \begin{compactitem}
        \item userrelevante Informationen, die angezeigt werden, sind: 
        \begin{compactitem}
            \item Name
            \item Profilfoto
            \item Erstellte Ausstellungen        
        \end{compactitem}
        \item Der*Die User*in kann dort seine Ausstellungen löschen.         
    \end{compactitem} 
    \item  Als User*in will ich bei der Erstellung einer Ausstellung zwischen verschiedenen Templates wählen können, um diese auf einfache Weise zu individualisieren. Auswahlkriterien: 
    \begin{compactitem}
        \item Es gibt 2 Templates am Anfang, in denen die Wandfarbe, der Boden und die Podeste für die Dateien vorgegeben sind.
        \item Der*Die User*in kann keine Templates erstellen.
        \item Der*Die User*in kann manuell die Podeste für jede Datei zwischen 5 vorgefertigten auswählen und für den Boden + Wand ein Pattern hochladen
        \item Bei den Templates kann man aber manuell einzelne Aspekte verstellen (siehe Punkt oben)
    \end{compactitem} 
    \item Als User will ich meine Daten auf den Server laden, um diese jederzeit innerhalb einer Ausstellung platzieren und integrieren zu können. Auswahlkriterien: 
    \begin{compactitem}
        \item Unterstützte Dateien (Bilder-, Audio-, Video-, 3D-Dateien) 
        \item Uploadmöglichkeiten: 
        \begin{compactitem}
            \item Drag and Drop
            \item Im Ordner auswählen.
        \end{compactitem} 
        \item Dateigrößenlimit: 50MB
        \item Dateityp Limit: nur Standardformate (Bilder: JPG, PNG, etc.;        Audio: WAV, MP3, etc.; …)
        \item Fehlermeldung bei falschem Upload
    \end{compactitem} 
    \item  Als User*in will ich, dass meine Daten automatisch als Ausstellungsstücke in der Ausstellung angeordnet werden, damit die Nutzung für persönliche Bereiche unkompliziert möglich ist. Auswahlkriterien: 
    \begin{compactitem}
        \item Falls dies nicht möglich ist, soll eine Fehlermeldung angezeigt werden
    \end{compactitem} 
    \item Als User*in möchte ich die Reihenfolge und Platzierung meiner Werke innerhalb der Ausstellung adaptieren können. Auswahlkriterien: 
    \begin{compactitem}
        \item Die Werke sollen zwischen vordefinierten Plätzen tauschbar sein.
    \end{compactitem} 
    \item Als User*in will ich andere Rechte haben, je nachdem ich angemeldet bin oder nicht. Auswahlkriterien: 
    \begin{compactitem}
        \item Wenn ein*e User*in angemeldet ist, kann er*sie:
        \begin{compactitem}
            \item Ausstellungen ansehen 
            \item Ausstellungen erstellen/löschen 
            \item Ausstellungen favorisieren
        \end{compactitem}  
        \item Wenn ein ein*e User*in nicht angemeldet ist, kann er*sie: 
        \begin{compactitem}
            \item Ausstellungen ansehen 
            \item keine Ausstellungen erstellen 
            \item keine Ausstellungen favorisieren
        \end{compactitem} 
    \end{compactitem} 
    \item Als User*in möchte ich meine Ausstellung abspeichern und löschen können. Auswahlkriterien: 
    \begin{compactitem}
        \item Die Daten im Bezug auf die Ausstellungen werden auf dem Server gespeichert.
        \item Bevor die Ausstellung gelöscht wird, soll ein Warnhinweis angezeigt werden, welcher noch bestätigt werden muss. 
    \end{compactitem} 
\end{compactenum}

\section{Evolutionäres Prototyping}
\label{ch::ongoing-prototyping}
Das evolutionäre Prototyping ist eine Art des Prototyping, dabei wird der Prototyp über die Dauer des gesamten Projektes weiterentwickelt, bis daraus das fertige Softwareprojekt wird.

In der Softwareentwicklung kommt es zu großen Problemen. Der Entwicklungsprozess dauert lang und die Kundschaft sieht erst am Ende des Werdegangs das funktionierende Produkt, deshalb kann es dazu kommen, dass Fehler und Missverständnisse aus der Spezifikationsphase erst zu diesem Zeitpunkt erkannt werden. Zu diesem Zeitpunkt ist jede Veränderung am Projekt sehr kostenaufwendig. Eine Lösungsstrategie für dieses Problem ist die Anwendung von Prototypen. Prototypen sind Annäherungen oder funktionelle Modelle von Systemteilen des fertigen Produkts. Sie werden im Vergleich zu dem Produkt mit weniger Aufwand produziert und das Klientel kann sich vorab ein Bild machen und Missverständnisse kommen durch die vermehrte Absprache mit der Kundschaft weniger auf und kann mit wenig Aufwand gelöst werden. \cite{Prototyping}

\section{Prototyp im Projekt}
Im Projekt wurden mehrere Prototypenarten verwendet, primär aber der evolutionäre Prototyp.

Anfangs wurde so die Machbarkeit des Projekts mit einer Umsetzbarkeitsanalyse, bei der der Prototyp eine große Rolle spielte, getestet. Im Prototyp (siehe Seite \pageref{ch:Technologien:AngularThree} Angular Three) wurde überprüft, ob die Technologien auf die das Projekt aufbaut auch die Anforderungen (siehe Seite \pageref{ch:umsetzung:projekt-user-stories} Projekt User Stories) erfüllen konnten. Dadurch konnten die Erkenntnisse gemacht werden, dass Angular Three nicht geeignete für das Projekt war und eine anderen 3D-Web-Api ThreeJs benutzt werden. Da dieses Resultat schon früh in dem Entwicklungsprozess gemacht wurde, kam mit der Änderung der 3D-Web-Apis im nur wenig Programmieraufwand zusätzlicher Aufwand.

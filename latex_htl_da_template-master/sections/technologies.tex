\section{Angular}

\subsection{Allgemeines}
\setauthor{Fabian Maar}
Angular ist ein Framework für Webapplikation, das auf der Programmiersprache Typescript basiert. Es ist eines der renommiertesten Frameworks zur Front-End-Entwicklung und wird als Open-Source-Software zur Verfügung gestellt. Besonders bietet sich Angular für Single-Page-Webanwendung an, da es ein komponentenbasiertes Framework ist. Das bedeutet, der Code ist wiederverwendbar und verkapselt. Komplexe Logiken werden auf ihre Grundelemente reduziert und beeinflussen sich nicht gegenseitig. 

\subsection{Vorteile/Auswahlkriterien}
\setauthor{Fabian Maar}
Dieses komponentenbasierte Programmieren war eines der vielen Gründe 
warum Angular die beste Option ist. Im Abschnitt 5 //TODO  wird nochmals deutlich, wie dieser Aspekt genutzt wird. Ein weiterer Grund, ist die einfache Einbindung von 
Libraries und deren Übersichtlichkeit, die durch ngModules ermöglicht werden. Die ständige Erweiterung und die Unterstützung von Third-Party-Libraries, die unter anderem für die 3D-Darstellung verwendet werden, wird ebenfalls von diesen ngModulen ermöglicht. 
Beispiele wie wir dieses ngModule aufgesetzt haben befinden sich im Abschnitt Landing-Page aufsetzen //TODO VERLINKEN. Auch aufgrund von persönlichen Erfahrungen im Unterricht und in anderen Projekten war Angular die beste Auswahlmöglichkeit.



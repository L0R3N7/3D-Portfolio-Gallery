\setauthor{Litzlbauer Lorenz}
In diesem Kapitel wird das Projekt noch einmal revue passiert, die Probleme und ihre Lösungen beschrieben, die Untersuchung der Anliegen abgeschlossen und mehrere Erweiterungsvorschläge gegeben für die Weiterentwicklung des Projektes.

\section{Probleme und Lösungsstrategie}
\subsection{Angular und Design Frameworks [L]}
\setauthor{Litzlbauer Lorenz}
Die erste Herausforderung war, dass Design von Angular-Material-Komponente anzupassen.

Das Angular Webframework kapselt eigene Komponenten ab, damit sie sich nicht gegenseitig mit ihren Stylingbefehlen beeinflussen. Angular verwendet für die Einkapselung eine virtuellen Dom oder auch eine, nativen vom Browser unterstützen, ShadowDOM-API (siehe \ref{glos:shadow-dom}).
\cite{AngularViewencapsulation}

Es gibt Situationen, in denen übergeordnete Komponenten das Styling untergeordneter ändern müssen. Besonders bei Angular Materials, einem UI- und Komponenten-Framework von Angular, kommt es oft dazu, dass die Angular-Material-Komponente meistens aus vielen kleinen Komponenten aufgebaut wird. Wenn nun der Style angepasst werden muss, damit die Material Komponente mehr zu der Corporate Identity des Produktes passt, ist das schwierig umsetzen.

Um dieses Problem zu lösen, gibt es zwei Möglichkeiten - \emph{::ng-deep} und das globale Stylesheet.

\subsubsection{::ng-deep [L]}
\emph{::ng-deep} durchbricht den virtuellen DOM und ShadowDOM und erlaubt es in Kombination mit weiteren Styleselektoren von der Eltern-Komponente auf alle Kinder-Komponenten zuzugreifen. (siehe Code Beispiel \ref{lst:sum:ngdeep})

\begin{lstlisting}[caption={{Parent.component.scss - Changing Styling in Child Componentes by using :ngdeep}},language=HTML,label=lst:sum:ngdeep]
::ng-deep app-child-component{
    -- change styling of child component
    background-color: pink;
}    
\end{lstlisting}

\emph{::ng-deep} funktioniert, indem es die Einkapselung der Komponente ausschaltet und jeden Selektor (in Kombination mit \emph{::ng-deep}) zu einem globalen Style befördert. Das kann, jedoch zu unvorhersehbaren Fehlern und Problemen führen. Empfehlenswert ist es, wenn man einen globalen Wert verändern will, das auch im globalen Stylesheet zu machen.
Wenn der*die Entwickler*in Zugriff auf die Komponente hat und sie ändern kann, gibt es keinen Grund \emph{::ng-deep} zu verwenden. Anpassungen können direkt in der Komponente gemacht, oder durch Input eine Interaktionsmöglichkeit hinzugefügt werden. Bei externen Komponenten wie Angular Material muss wohl eine Style-Überschreibung mittels \emph{::ng-deep} oder dem globalen Stylesheet erfolgen, weil der*die Entwickler*in keinen Zugriff auf die Komponente hat. \cite{AngularComponentsStyleNgDEEP, UnderstandingNgDeep}

\subsection{Schnittstellen [Team]}
Ein großes Problem bei der Entwicklung war die späte Fertigstellung des Servers.
Dadurch konnten Funktionen der Single-Page-Application nicht mit realen Daten getestet werden. Der Lösungsversuch war es, Mockup-Daten zu verwenden und damit die Daten des Servers zu simulieren.  

Zusammenfassend kann gesagt werden, dass dies keine gute Lösung darstellte. Arbeitsschritte mussten mehrere Male erledigt werden, wodurch sich der Aufwand erhöhte. Besser wäre, eine bessere Kommunikation mit der Backend-Entwicklung aufrechtzuerhalten und das Backend und Frontend schnellstmöglich zu verbinden. 

\section{Zielerreichung [Team]}
Die im Projekt angestrebten Ziele, Untersuchungskriterien, User Stories und Meilensteine wurden erfüllt.

Das Endergebnis bietet Besucher*innen der Webseite eine Plattform, auf der sie sich und ihre Kunst präsentieren und sich informativ weiterbilden können. Das Produkt besteht aus insgesamt drei Kernbereichen: Der Webseite, der Ausstellung und dem Content Creation Tool.

Auf der Startseite des Web Applikation wird dem*der Benutzer*in das Projekt vorgestellt und zum Mitmachen angeregt. Des Weiteren wurde ein User Management System erstellt, bei dem sich Benutzer*innen registrieren und anmelden können. Dabei wurde auf die Sicherheit der Benutzer*innen geachtet. Außerdem ist es möglich, nach beliebigen Ausstellungen und Kategorien zu suchen, wobei ebenfalls die neuesten Ausstellungen empfohlen werden.

Das Content-Creation Tool ermöglicht es dem*der Benutzer*in einfach eine Ausstellung zu kreieren. Der*Die Benutzer*in lädt seine*ihre Ausstellungsobjekte einfach hoch, wählt einen vordefinierten virtuellen Raum aus und konfiguriert diesen. Danach kann die Ausstellung auf der Webseite veröffentlicht werden.

In der Ausstellung werden Ausstellungsobjekte vom Server dynamisch geladen und aufgebaut. Für den*die Benutzer*in wurden verschiedene Bewegungsarten und Interaktionsmethoden innerhalb des Ausstellungsraumes implementiert.

\section{Erweiterungsvorschläge [Team]} 
Das Projekt 3D-Gallery-Portfolio soll weiterentwickelt werden. Es ist auf Github öffentlich verfügbar, wodurch die Open-Source-Community es aktualisieren und erweitern kann.
\section{Cinema 4D [M]}
\setauthor{Fabian Maar}
Cinema 4D ist ein professionelles Programm vom Unternehmen Maxon zur 3D-Modellierung und Animation. Zum einen ist die Software leicht zu bedienen und zum anderen bietet Maxon eine Vielzahl an Tutorials und Anleitungen. Auch aufgrund von persönlichen Erfahrungen im Unterricht wurde das Programm zur Erstellung von diversen 3D-Modellen benutzt. Schon bereits in der Vergangenheit modellierte Werke konnten sich zu Nutze gemacht und teilweise auch im 3D-Raum verwendet werden. Ein weiterer wichtiger Aspekt ist, dass Cinema 4D den Export von GLTF-Files unterstützt. GLTF ist das Dateiformat, das genutzt wird, um ein 3-Modell in der Three.js Szene zu laden. Da ähnliche Programme wie zum Beispiel Blender diese Exportmöglichkeit nicht anbieten, war Cinema 4D die gewählte Option für die 3 Modellierung. 
\cite{Cinema4D}
 
\section{Figma [L]}
\setauthor{Litzlbauer Lorenz}
\label{ch::technologies::figma}
Figma ist ein Programm für die Erstellung und Testung von UI und UX Prototypen. Das Programm ist einsteigerfreundlich, denn es hat eine benutzerfreundliche Oberfläche und es gibt viele Tutorials auf der Webseite und von der Figma Community. Figma ist primär eine Web-Applikation, bietet aber auch eine Desktop-Version für macOS und Windows und eine mobile Version für iOS und Android an. Figma arbeitet mit mehreren Konzepten, um den Designprozess zu vereinfachen:
\paragraph{Kollaboration}
In Figma können mehrere Personen gleichzeitig auf verschiedenen Geräten designen. Es gibt verschiedene Rollen, wie Zuschauer*in, Kritiker*in oder Mitarbeiter*in. Die Möglichkeiten können genutzt werden um einen*einer Kunden*in in den Designprozess zu involvieren und schon früh Feedback auf das Design bekommen zu können.
\paragraph{Plugins}
Figma hat eine große Community, gibt es ein Feature nicht, kann dieses von der Community im PluginStore als Plugin hinzugefügt werden. Ein Plugin ist eine Software-Erweiterung, welche die Fähigkeiten oder die Features eines Softwareprojektes erweitert.

\paragraph{Assets}
Figma arbeitet mit dem Konzept der Assets. Bereits designte Komponenten können als Assets gespeichert werden. Figma legt einen großen Wert auf die Modularität, es können Farben und Pixelgrößen als Variable gespeichert werden, wenn diese sich verändern, verändern sich gleichzeitig die darauf referenzierenden Assets.

\subsection{Auswahl}
Wegen dieser vielen Features und persönlichen Erfahrungen in privaten Projekten bot sich Figma für dieses Projekt als UX/UI-Design- Tool an.
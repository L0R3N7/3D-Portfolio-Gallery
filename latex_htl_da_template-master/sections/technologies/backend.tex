\subsection{Applikationserver}
Als Applikationsserver wurde Quarkus ausgewählt. Quarkus zeichnen sich aus durch kurze Startzeiten und gute Performance in Hinsicht auf den Verbrauch des Arbeitsspeichers. Nach der Erstellung eines neuen Projekts wird standardmäßig eine Maven-Struktur mitgeliefert. Zusätzlich verfügt Quarkus eine Vielzahl von Extensions, welche durch Command-line-Befehle oder händisch zu Projekten hinzugefügt werden können. \cite{QuarkusAbout, QuarkusFirstApplication}
Quarkus besitzt die Fähigkeit im Hintergrund nach jeder Sourcecodeänderung die betroffenen Applikationen upzudaten und Unittest auszuführen.
Die Services werden als REST-Services bereitgestellt. Quarkus besitzt eine einfache Konfigurationsmöglichkeit um REST-Services anzubinden. 

\subsection{Datenhaltung}
PostgreSQL ist ein open source Managementsystem für relationale Datenbanken, welches seit 35 Jahren entwickelt wird. Hinter diesem System befindet sich eine große Community, welche weiterhin Features entwickelt. Die Entscheidung Eine gute Dokumentation und die vielen verschiedenen Anwendungsfälle \cite{PostgreSQLAbout}






%%\section{Grundkriterien für das Backend [E]}
%%\setauthor{Halilovic Ema}
%%Die Grundkriterien für die Wahl der Tools und Technologien hier waren, dass diese fortlaufend weiterentwickelt werden 
%%und zum derzeitigen Standpunkt ein ausreichendes Spektrum an Funktionalität für Mikroservice-Architekturen ermöglichen.\cite{MicroserviceAbout} 
%%Ebenso sollen diese gute Dokumentationen und breite Communities enthalten. Schon vorhandene Praxiserfahrung stellte sich bei der finalen Auswahl als entscheidender Faktor dar.
%%
%%
%%
%%\section{PostgreSQL [E]}
%%\setauthor{Halilovic Ema}
%%PostgreSQL ist ein open source Managementsystem für relationale Datenbanken, welches seit 35 Jahren entwickelt wird. 
%%Hinter diesem System befindet sich eine große Community, welche weiterhin Features entwickelt. 
%%Die Entscheidung  Eine gute Dokumentation und die vielen verschiedenen Anwendungsfälle 
%%\cite{PostgreSQLAbout}
%%
%%\section{Quarkus [E]}
%%\setauthor{Halilovic Ema}
%%Unter Quarkus versteht man ein open source Framework, womit cloud-native Projekte in Java entwickeln werden können. 
%%Dieses zeichnen sich aus durch kurze Startzeiten und gute Performance in Hinsicht auf den Verbrauch des Arbeitsspeichers. 
%%Nach der Erstellung eines neuen Projekts wird standardmäßig eine \hyperref[ch::MavenTool]{Maven}-Struktur mitgeliefert.
%%Zusätzlich verfügt es eine Vielzahl von Extensions, welche durch Command-line-Befehle oder händisch zu Projekten hinzugefügt werden können.
%%
%%\cite{QuarkusAbout, QuarkusFirstApplication}
%%
%%\subsection{Maven [E]}
%%\label{ch::MavenTool}
%%\setauthor{Halilovic Ema}
%%Maven ist ein Tool, um den Kompilierungsprozess eines Projekts zu vereinfachen. 
%%Wenn man Projekte auf unterschiedlichen Geräten ausführen möchte kommt es dabei oftmals zu Problemen. Meist sind lokale Konfigurationen der Auslöser dafür.
%%Maven ermöglicht ebenso ein einheitliches System für Projektkonfigurationen, sodass diese nicht mehrmanuell bei Gerätewechsel umgestellt werden müssen. 
%%Dabei wird die \emph{pom.xml} Datei verwendet, welche eine der Hauptkomponenten ist für Maven-Projekte.
%%\cite{MavenAbout}
%%
%%In Quarkus-Projekten werden darin relevante Informationen gespeichert, wie zum Beispiel die verwendete Java Version oder alle verwendeten Extensions.
%%\
%%
%%\subsection{JDBC Driver - PostgreSQL [E]}
%%\setauthor{Halilovic Ema}
%%JDBC Driver - PostgreSQL ist eine Extenstion für Quarkus Projekte, die Datenbankverbindungen zu PostgreSQL Datenbanken ermöglicht.
%%Java JDBC gibt es als API für Java Anwendungen, jedoch unterschiedet diese sich von dem in Quarkus benutzten JDBC Driver.
%%
%%// TODO alles umschreben
%%Das JDBC steht für \emph{Java Database Connectivity} und .
%%
%%Um eine Datenbankverbindung aufzubauen muss man in die Datei \emph{application.properties} einige zusätzlichen Konfigurationen einfügen, wie den Pfad der Datenbank, die Art der Datenbank, den Nutzernamen und das Passwort \ref{lst:quarkusDatasource}:
%%
%%\begin{lstlisting}[caption=Beispielkonfigurationen,label=lst:quarkusDatasource]
%%  quarkus.datasource.db-kind=postgresql 
%%  quarkus.datasource.username=meinUser
%%  quarkus.datasource.password=meinPassword
%%  quarkus.datasource.jdbc.url=jdbc:postgresql://localhost:5432/meineDatabase
%%\end{lstlisting}
%%
%%\subsection{Hibernate ORM [E]}
%%\setauthor{Halilovic Ema}
%%// TODO
%%
%%\subsection{REST-Easy [E]}
%%\setauthor{Halilovic Ema}
%%REST-Easy ist eine Erweiterung für Quarkus, die es ermöglicht, im Quarkus Projekt mit Jakarta RESTful Web Servies zu arbeiten. 
%%
%%
%% \section{IntelliJ IDEA [E]}
%% \setauthor{Halilovic Ema}
%% IntelliJ IDEA ist eine IDE, welche von JetBrains entwickelt wurde. Diese ist ausgelegt für Java- und Kotlin-Projekte. 
%% Durch eingebaute Features erleichtert diese Entwicklungsumgebung das Programmieren für den*die Nutzer*in. 
%% Plug-Ins ermöglichen es, Datenbankverbindungen und weiteres in der IDE zu konfigurieren, sodass dem*der Entwickler*in eine Übersicht von benötigten Informationen gegeben werden kann.
%% \cite{IntelliJIDEA}
%%
%%\section{Oracle Server [E]}
%%\setauthor{Halilovic Ema}
%%// TODO
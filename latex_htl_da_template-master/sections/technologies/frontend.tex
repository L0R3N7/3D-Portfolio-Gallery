\subsection{Allgemeines [M]}
\setauthor{Fabian Maar}
Angular ist ein Framework für Webapplikation, das auf der Programmiersprache Typescript basiert. Es ist eines der renommiertesten Frameworks zur Front-End-Entwicklung und wird als Open-Source-Software zur Verfügung gestellt. So besitzt Angular eine große Community und wird von vielen Software-Entwicklern benutzt. So zählt es 2022 zu dem zweitbeliebtesten Front-End Framework \cite{AngularEvidence} und wird von Milliardenkonzernen wie zum Beispiel Google oder PayPal verwendet. 
\cite{AngularEvidence2}

Die große Stärke von Angular ist die Erstellung von Single-Page-Webanwendung, da es eine komponentenbasierte Struktur besitzt. Das bedeutet, der Code ist wiederverwendbar und verkapselt. Komplexe Logiken werden auf ihre Grundelemente reduziert und beeinflussen sich nicht gegenseitig.
\cite{AngularGeneral}

\subsection{Dependencies [L]}
\setauthor{Litzlbauer Lorenz}
In den folgenden Unterabschnitten wird sich mit den Technologien auf die Angular aufgebaut ist, auseinandergesetzt. 
\subsubsection{RxJS}
ReactiveX ist eine Library für das Erstellen von asynchronen und Event-basierenden Programmen, dafür benutzt es  \emph{observable sequences}.
Die Library erweitert so, dass \emph{Observer-Design-Pattern} mit verschiedenen neuen Operatoren und händelt dabei "Lowlevel"-Funktionen wie Threading, Synchronisation, Thread-Sicherheit und das nicht-blockieren von I/O vergleiche \cite{ReactiveXIntro}

RxJS ist eine Implementierung von ReactiveX für die Programmiersprache Javascript. Angular verwendet RxJS für reaktive Programmierart.
\subsubsection{Webpack}
\setauthor{Litzlbauer Lorenz}
Die Hauptfunktion von Webpack ist es, viele verschiedenen Daten zu einem Paket für eine JavaScript Applikation zusammenzufassen, dabei optimiert es die Dateigröße, indem es mithilfe von einem selbst generierten \emph{dependency-graph} die Abhängigkeiten der Applikation überprüft und nur notwendige Teile für die JavaScript-Applikation der Daten nimmt. Bei der Zusammenfassung werden Sprachen, die der Webbrowser nicht Interpretieren kann wie Typescript, Sass uvm. in die für den Webbrowser verständlichen Gegenstücke gewandelt. Webpack hat aber auch viele andere Features. Vergleiche \cite{Webpack}
\setauthor{Litzlbauer Lorenz}
Angular benutzt Webpack um TypeScript in JavaScript und Sass bzw. Scss in Scss zu wandeln, um beim Bauen des Projektes alle Module in ein einziges zusammenzufassen und um die Applikation bei der Entwicklung zu \emph{Live-Reloading} zu unterstützen. Dabei wird bei einer Änderung im Code die gesamte Applikation geupdatet und neu gestartet. 

\subsection{Vorteile/Auswahlkriterien [M]}
\setauthor{Fabian Maar}
Folgende Aspekte sprechen für die Nutzung Angular in diesem Projekt:

\begin{itemize}
  \item komponentenbasiertes Programmieren
  Im Abschnitt 5 //TODO wird nochmals deutlich, wie dieser Aspekt genutzt wird.
  \item einfache und übersichtliche Einbindung von Libraries durch ngModules; wird behandelt in Kapitel NgModules \ref{sec:NgModules} 
  \item Isolierung der Logik mit der Benutzeroberfläche durch die Model-View-Controller Architektur und die Model-View-Viewmodel Architektur; wird behandelt in Kapitel MVC und MVVM \ref{sec:MVCandMVVM} 
  \item 
  Dependency Injection für die Verbindung von Front-End zum Back-End //TODO siehe Kapitel HTTP Services
\end{itemize}

\subsection{Model-View-Controller und Model-View-ViewModel Architecture [M]}\label{sec:MVCandMVVM}
\setauthor{Fabian Maar}
\subsubsection{Allgemein}

Angular basiert auf dem Konzept des Model-View-Controller- (MVC) und Model-View-ViewModel (MVVM) Architecture Patterns. Diese Patterns werden verwendet, um die Logik von der Benutzeroberfläche zu trennen und komplexe Aufgaben einfacher zu bewältigen. Dies wird in der Fachsprache auch oft als "Separation of Concerns” bezeichnet.
\cite{AngularArchitecturePattern}

Architecture Patterns sind Muster die verwendet werden um:
\begin{itemize}
  \item eine Software-Anwendung zu strukturieren
  \item die Redundanz von wiederholenden Code-Teile zu vermeiden 
  \item immer wiederkehrende Probleme durch eine einmalige Lösungen zu beheben
  \item um die Wartung und Testung zu erleichtern
  \item Änderungen am Umfang und der Größe der Applikation leichter handhaben zu können
\end{itemize}

\cite{MVCmdn}
\cite{MVVM}
\cite{MVC}

\subsubsection{Model-View-Controler (MVC)}
Die "Separation of Concerns” wird bei diesem Pattern durch die Aufteilung der Applikation in Model, View und Controller realisiert. Dadurch ist es möglich, sich auf einen Aspekt der Implementierung zu fokussieren. 

\begin{figure} [h t]
  \centering
  \includegraphics[scale=0.5]{pics/mvc.png}
  \caption{Die Model-View-Controller Pattern \cite{MVCmdn}}
  \label{fig:tech:front:mvc-architecture}
\end{figure}

Welche Daten eine Applikation beinhalten soll, wird über das Model geregelt. Falls sich das Model ändert, werden die Benutzerfläche und manchmal auch der Controller entsprechend geändert. In Angular sind diese Modelle mit den Interfaces //TODO REFERENZIEREN  zu vergleichen.

Die View ist die Benutzeroberfläche des Programms. Der*die Benutzer*in kann mit dieser interagieren und diese verändern. Sie wird aus den Daten des Models erstellt und definiert. In Angular kann diese View mit dem HTML-Template einer Komponente verglichen werden.

In dem Controller werden die Eingaben der Benutzer*innen verarbeitet und die betroffenen Model- und View-Komponenten beeinflusst und verändert. Auch ist es möglich zu bestimmen, welche Views verändert werden sollen und die Daten gegebenenfalls in unterschiedliche Formate anzuzeigen. Die TypeScript-Files in Angular sind vergleichbar mit diesen Controllern.
\cite{MVC}

\subsubsection{Model-View-ViewModel (MVVM)}
Diese Architektur basiert auf dem Konzept des MVC-Patterns. Das MVVM realisiert die “Separation of Concerns” durch die View, ViewModel und Model Komponenten.

\begin{figure} [h t]
  \centering
  \includegraphics[scale=0.5]{pics/mvvm-pattern.png}
  \caption{Die Model-View-ViewModel Pattern \cite{MVVM}}
  \label{fig:tech:front:mvc-architecture}
\end{figure}

Die View funktioniert hierbei ähnlich wie beim MVC-Pattern, jedoch ist es möglich, dass eine View auch Logik enthält, die Änderungen am Aussehen durchführt. Daher ist das HTML-Template einer Angular-Komponente noch ähnlicher zu dem Konzept einer View aus dem MVVM-Pattern als zu einer View aus dem MVC-Pattern. 

Im ViewModel wird die Funktionalität der Benutzeroberfläche bestimmt. Dabei informiert das ViewModel die View über Änderungen im Model und beliefert es mit Daten. Dieser Vorgang beschreibt das Konzept des Data-Bindings //TODO REFERENZ in Angular.  

Das Model funktioniert im Grunde gleich wie beim MVC-Pattern, nur wird hierbei nicht die View, sondern das ViewModel über Änderungen informiert. Somit ist das ViewModel das Mittelstück zwischen View und Model.

Angular nutzt hierbei eine Mischung aus beiden Architecture-Patterns. Sie wird durch das Konzept von Komponenten realisiert, auf welches im nächsten Abschnitt näher eingegangen wird. Dabei kann man die Parallelen der Patterns erkennen.
\cite{MVVM}

\subsection{NgModules [M]}\label{sec:NgModules}
\setauthor{Fabian Maar}
NgModules erleichtern die Einbindung von Libraries enorm. Da man sich keine externen Files mühsam downloaden, einbinden muss und jeder Import-Prozess fast von selbst geschieht, erspart man sich enorm Zeit beim Entwickeln. Außerdem sind alle Importe in chronologischer Reihenfolge gelistet, wodurch ein guter Gesamtüberblick geliefert wird und schnell überflüssige Imports entfernt werden können. Die ständige Erweiterung und die Unterstützung von Third-Party-Libraries, die unter anderem für die 3D-Darstellung verwendet werden, wird ebenfalls von diesen ngModulen ermöglicht. Die Angular-Applikation wird hierbei organisiert und gestartet in dem die Metadaten wie folgt abgespeichert werden:

\begin{itemize}
  \item Declarations - In dieser Sektion werden alle zugehörigen Komponenten, Direktiven und Pipes deklariert 
  \item Providers - Sie initialisieren wie die Werte die bei der Dependency Inection //TODO REFERENZ abgerufen werden \cite{AngularProviders}
  \item Imports - Hier werden alle Libraries und exportierte Modules importiert
  \item Exports - Sie beinhalten alle zu exportierenden Module
  \item Bootstrap - Hierbei wird angegeben welche Komponente beim Anwendungsstart zuerst geladen wird
\end{itemize}

Wie die ngModules verwendet in der 3D-Gallerie-Applikation wurden, wird im Abschnitt Routing oder Landing Page aufsetzen //TODO Referenz nochmals erklärt.
\cite{AngularNgModules}
\cite{AngularNgModulesAPI}
\cite{AngularBuch}


\section{Webstorm [M]}
\setauthor{Fabian Maar}
Webstorm ist eine Entwicklungsumgebung vom Unternehmen JetBrains, die sich auf die Programmiersprache JavaScript spezialisiert hat. Sie wurde besonders für das Arbeiten mit Angular optimiert. Dies zeigt sich durch viele Features, die das Entwickeln von Angular-Projekten erleichtern. So macht es Webstorm möglich, mit nur wenigen Mausklicks eine neue Angular-Dependency oder Komponente zu erstellen. Auch wird die Entwicklungszeit durch intelligente Code-Vervollständigung, Code-Formatierung, einfache Navigation und viele weitere hilfreiche Features deutlich verkürzt.


\section{ThreeJs [L]}
\setauthor{Litzlbauer Lorenz}
\begin{wrapfigure}{r}{0.3\textwidth}
    \begin{center}
      \includegraphics[width=0.2\textwidth]{pics/threeJS.png}
     \caption{ThreeJs Logo}
    \end{center}
\end{wrapfigure}
ThreeJs ist eine JavaScript Library für die Darstellung von 3D-Grafiken im Web. Für die 3D-Darstellung nutzt ThreeJs meistens WebGL (mehr dazu im nächsten Abschnitt \hyperref[ch::ThreeJsDependency]{ThreeJs Dependency}), ein low-level Framework. WebGl hat eine hohe Komplexität. ThreeJs bietet eine Abstraktion, bei der es viele Sachen wie die 3D-Szene, Lichter, Schatten, Materialien, Texturen und 3D-Mathe händelt, um die 3D-Darstellungen im Web einfacher zu gestalten. In ThreeJs werden Geometrie, Objekte und Materialien verbunden, um ein 3D-Objekt zu erstellen. Dabei kann die Struktur einer Szene der Abbildung \ref{fig:tech:front:threejsstructure} ähneln. Vergleiche \cite[ThreeJs fundamentals]{ThreeJsFund}

\begin{figure} [h t]
    \centering
    \includegraphics[scale=0.5]{pics/threejs-structure.png}
    \caption{Die Struktur von ThreeJs}
    \label{fig:tech:front:threejsstructure}
\end{figure}

Abbildung \ref{fig:tech:front:threejsstructure} \cite{ThreeJsFund}

\subsection{Dependency}
\label{ch::ThreeJsDependency}
Um einen Großteil der 3D-Darstellungen zu rendern, benutzt ThreeJs WebGL.

\subsubsection{WebGL}
\label{ch::webgl}
\begin{wrapfigure}{h  r}{0.3\textwidth}
    \begin{center}
      \includegraphics[width=0.2\textwidth]{pics/WebGL_Logo.png}
     \caption{ThreeJs Logo}
    \end{center}
\end{wrapfigure}
WebGl ist eine lizensfreie Api, die dafür benutzt wird 3D-Grafiken im Web darzustellen. Sie basiert auf OpenGL ES 2.0 und benutzt auch dieselbe shading language GLSL. WebGL ist eine Low-Level-API, das bedeutet, dass schon sehr einfache Projekte einen hohen Programmieranteil haben. Vergleiche\cite[WebGl Getting Started]{WebglGettingStarted}


\subsection{Auswahlkriterien}
Es gab verschiedene Auswahlkriterien für die 3D-Web-API.
\begin{compactitem}
  \item Effizienzen der 3D Engine (Hardware Acceleration, RAM-Auslastung)
  \item Benutzerfreundlichkeit (Programmerexperience)
  \item Zusätzliche Features
  \item Die Features des Projektes müssen damit umsetzbar sein
  \begin{compactenum}
    \item Das Laden von 3D-Modellen aus 3D-Dateinen und aus dem Internet
    \item Video-Texturen support
  \end{compactenum}
\end{compactitem}

\subsubsection{Alternativen 3D Web Apis}
Es gibt viele Technologien, die 3D-Grafiken im Web ermöglichen, wie Three.js, Babylon.js, A-Frame, X3DOM und WebGL

\paragraph{A-Frame}
A-Frame wird von der Morzilla Foundation als OpenSource-Projekt entwickelt. Die 3D-Szene wird durch eine deklarative Sprache mit XML-Syntax definiert. Über die WebVR-API bietet die Libray auch die Möglichkeit, die 3D-Szenen durch eine VR-Brille zu erfahren. Bei der 3D-Darstellung setzt A-Frame auf ThreeJs. \cite[A-Frame Wikipedia]{a-frame-wiki}

A-Frame hat ähnliche Funktionen und Leistung im Vergleich zu ThreeJs, da es ja schlussendlich darauf aufbaut. A-Frame sticht bei der VR support hervor (ThreeJs hätte auch eine Unterstützung dafür, diese ist aber schwieriger einzubinden), doch ist das kein unbedingt nötiges Feature. ThreeJs ist ja bereits eine Abstraktion von WebGL. Für das Projekt wird keine weiter Abstraktion gebraucht.

\paragraph{WebGL}
Im Vorherigen Kapitel wurde sich bereits mit WebGL befasst. \ref{ch::webgl}
Um nur eine einfache 3D-Szene in WebGL darzustellen, muss sehr viel Code geschrieben werden. Denn WebGL übernimmt keine Low-Level-Funktion. Es müssen Matrixrechnungen für die Transformationen von 3D-Objekt Manuel und Vertex buffers, die die Daten der Vertex Positionen, Normaldaten, Farben und Texturen, selbst programmiert werden und das in einer eigenen Programmiersprache. Deshalb war WebGL keine Option für das Projekt.

\subsubsection{Angular Three}
\label{ch:Technologien:AngularThree}
Angular Three ist ein Open Source Projekt von Matt DesLauriers. Es zielt darauf ab die Vorteile von Angular und ThreeJs zu kombinieren. Dabei verbindet es das Prinzip der Komponenten von Angular mit der 3D Darstellung von ThreeJs. 

\begin{lstlisting}[language=html,caption=Angular Three - Komponentenbasiertes 3D Scenen in HTML,label=lst:impl:AngularThreeExampleCode]
<ngt-canvas>
    <ngt-ambient-light intensity="0.5"></ngt-ambient-light>
    <ngt-spot-light [position]="10" angle="0.15" penumbra="1"></ngt-spot-light>
    <ngt-point-light [position]="-10"></ngt-point-light>
  
    <app-cube [position]="[1.2, 0, 0]"></app-cube>
    <app-cube [position]="[-1.2, 0, 0]"></app-cube>
  
    <ngt-soba-orbit-controls></ngt-soba-orbit-controls>
</ngt-canvas>
\end{lstlisting}

\begin{lstlisting}[language=html,caption=Angular Three - App Cube,label=lst:impl:AngularThreeCube]
<ngt-mesh
  (beforeRender)="onCubeBeforeRender($event)"
  (click)="active = !active"
  (pointerover)="hovered = true"
  (pointerout)="hovered = false"
  [scale]="active ? 1.5 : 1"
  [position]="position"
>
  <ngt-box-geometry></ngt-box-geometry>
  <ngt-mesh-standard-material [color]="hovered ? 'turquoise' : 'tomato'"></ngt-mesh-standard-material>
</ngt-mesh>
\end{lstlisting}

Code Beispiele \ref{lst:impl:AngularThreeExampleCode} \ref{lst:impl:AngularThreeCube} \cite{AngularThreeDocumentationFirstScene}

Ein Vorteil von Angular Three ist, dass durch nur wenige Zeilen Code \ref{lst:impl:AngularThreeExampleCode} eine 3D-Szene mit Lichtern und Orientierungsfunktionen erstellt werden kann und die Business-Logik, App-Cube \ref{lst:impl:AngularThreeCube} durch die Verwendung einer Komponente ausgelagert werden kann.

Ein weiterer Vorteil von Angular Three ist die ausführliche Dokumentation mit Codebeispielen. \href{https://angular-three.netlify.app/docs/getting-started/overview}{(link) Angular Three Dokumentation https://angular-three.netlify.app/docs/getting-started/overview}

Wegen der vielen Vorteile hohe Benutzerfreundlichkeit, ähnliche 3D-Leistung zu ThreeJs (Angular Three basiert auf ThreeJs, welches selbst die WebGL-Renderengine benutzt) und wegen der Verbindung von Angular und ThreeJs Features bietet sich die Liberay für das Projekt an.

Mithilfe eines Prototypen ( mehr dazu im Abschnitt \hyperref[ch::ongoing-prototyping]{fortlaufendes Prototyping} ) wurde getestet, ob Angular Three alle Features, die für das Projekt benötigt werden, unterstützen kann, die für das Projekt nötig sind. Dabei wurde festgestellt, dass sich die Liberay Angular für das Projekt nicht eignet. Denn beim Prototyping sind Fehler aufgetreten. Beim erneuten Laden von 3D-Objekten aus derselben Datei haben sich keine Daten mehr im Objekt befunden. ThreeJs Module funktionierten Teilweise nicht, wie die FristPersonControls. 

Weil Angular Three nicht die für das Projekt aufgestellten Anforderungen erfüllt hat, wurde der Prototyp zu ThreeJs migriert, weil sich die Syntax ähnelt und dadurch nur wenig Aufwand bei der Migrierung entstand. Diese 3D-Library wurde schlussendlich auch im Projekt verwendet.
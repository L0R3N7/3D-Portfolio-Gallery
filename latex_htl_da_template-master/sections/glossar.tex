\subsubsection{Framework [M]}
\setauthor{Fabian Maar}

Ein Framework soll dem*der Programmierer*in helfen neue Applikationen zu schaffen. Es bietet somit ein Grundgerüst, auf dass sich die Software aufbauen lässt.  

\subsubsection{Open-Source [L]}
\setauthor{Lorenz Litzlbauer}
Open Source bei einer Applikation bedeutet, dass der Code dieser öffentlich zugänglich ist. Der Open Source Status sagt aber nichts über die Lizenzierung der Applikation aus. 

\subsubsection{Single-Page-Webanwendung[M]}
\setauthor{Fabian Maar}
Eine Single-Page-Webanwendung interagiert mit dem*der User*in, wodurch sich einzelne Komponenten der Website dynamisch verändern. //TODO

\subsubsection{Design-Pattern [L]}
\setauthor{Lorenz Litzlbauer}
Im Programmieren gibt es Probleme, die immer wieder auftreten. Diese Probleme wurden schon einmal sehr effizient gelöst und es besteht kein Nutzen sie immer wieder zu lösen. Design Patterns sind Strategien, wie man diese Probleme lösen kann. 

\subsubsection{Framework[M]}
\setauthor{Fabian Maar}
Event-Listener
//TODO

\subsubsection{Framework[M]}
\setauthor{Fabian Maar}
Third-Party [M]
//TODO

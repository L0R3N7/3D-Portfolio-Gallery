\setauthor{Litzlbauer Lorenz}
\section{Ausgangslage [L]}
Ausgangslage ist es, dass virtuelle Welten bzw. Realitäten sich derzeit einer großen Beliebtheit erfreuen. Viele Unternehmen wie Ikea oder Audi setzen auf virtuelle begehbare Ausstellungsräume, um ihre Produkte interaktiv zu präsentieren. \cite{VrVW}

Wir wollen einen Prototypen einer webbasierten virtuellen Ausstellung für Designer erstellen. In dieser Ausstellung sollen dreidimensionale Objekte, Fotos und Videos präsentiert werden können. Um unser Projekt weiter zu konkretisieren, überlegten wir uns die Anforderung, die sich an einen Designer von digitalen 3D-Objekten bei der Präsentation von 3D-Objekten stellen.

Die virtuelle Präsentation in einer 3D-Portfolio-Gallery ermöglicht beispielhaft zusätzliche Funktionen:
\begin{compactitem}
    \item Die Objekte besitzen eine vordefinierte genormte Größe. Eine Verkleinerung oder Vergrößerung erzielt eine gesteigerte optische Wirkung oder eine verminderte optische Wirkung.
    \item Die Gestaltung und Ausstattung von virtuellen Ausstellungsräumen ist mit wenigen Mausklicken zu ändern.
\end{compactitem}


\section{Themenstellungen [L]}
Die Diplomarbeit untersucht aktuelle Website-Technologien zur Darstellung von 3D-Modellen in Web-Browsern.
Dem Anwender soll eine interaktive grafische Benutzerschnittstelle zur Verwaltung und Gestaltung einer 3D-Gallery angeboten werden.
Im Frontend ist es Ziel, eine möglichst ansprechende Designsprache durch die Anwendung beizubehalten.
Daraus definieren sich folgende Anliegen:
\begin{compactitem}
    \item Auswahl und Entwicklung eines Designs für den Prototypen
    \item Bereitstellung einer Suchfunktion zum Auffinden von Ausstellungen
    \item Die Schaffung eines Konfigurationsmoduls für die Erstellung einer 3D-Gallerie. Die Konfiguration beinhaltet die Auswahl des Grundrisses der Gallery, die Platzierung der Objekte und den Upload von Multimediadaten.
\end{compactitem}


\section{Zielsetzung [L]}
In der vorliegenden Diplomarbeit ist das Hauptaugenmerk auf die technische Umsetzung gerichtet. Ziel ist ein funktionierender Prototyp.

Ein funktionierender Prototyp erfordert die Erstellung eines Frontends für die Benutzerinteraktion, eines Backends zur Datenhaltung und weiters eine Benutzerverwaltung und Benutzerauthentifikation. Die aktuell verfügbaren Softwareframeworks werden nach den Kriterien Verfügbarkeit und Einsetzbarkeit und Stand der Technik zur Erstellung des Prototypens untersucht und ausgewählt. Vorgegeben ist eine Client-Server-Architektur, da das fertige Softwareprodukt über einen Web-Browser im Internet angeboten wird und die Daten dezentral abgespeichert werden.

\section{Aufgabendifferenzierung [L]}
Das Projekt wird von zwei Personen bearbeitet. Die Projektleitung liegt bei Lorenz Litzlbauer. 

\subsection{Lorenz [L]}
Im Tätigkeitsbereich von Lorenz Litzlbauer wurde in der Softwareentwicklung das 
Usermanagment, 
das Tokenmanagment,
die Content Creation,
das UI-Design, 
die Platzierung von Objekten im Ausstellungsraum erstellt.

\subsection{Fabian [M]}
Im Lead von Fabian Maar wurde in der Softwareentwicklung die Such-Unterseite, Besucheransicht der Ausstellung, die Interaktion mit den Ausstellungsstücken, die Navigation im dreidimensionalen Raum und das UI-Design erstellt.
	

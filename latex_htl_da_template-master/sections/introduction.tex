\setauthor{Litzlbauer Lorenz}
\section{Ausgangslage}
Derzeit gibt es keine Möglichkeit für Designer, eine ansprechende Ausstellung im Internet zu erstellen, in welcher man verschiedene Dateiformate wie Audio-, Bild-, Video-, 3D-Dateien darstellen kann.

\section{Untersuchung Anliegen der individuellen Themenstellungen}
Untersuchen, welche Webseiten-Technologien sich für die Darstellung von 3D - Modellen eignen. Mit dieser Technologie auseinandersetzen und eine userinteractive Webseite gestalten, welche benutzerfreundlich umgesetzt werden soll.

Dieses Anliegen lässt sich in folgende Teilbereiche unterteilen:

Untersuchen...
\begin{compactitem}
    \item eines geeigneten Suchalgorithmus um andere Ausstellungen zu finden
    \item einer geeigneten Art, um mit der 3D-Web-Ausstellung zu interagieren bzw. sich zu orientieren.
    \item einer geeigneten Art der Content Creation einer 3D-Web-Ausstellung.
    \item der geeigneten Software für die Continuous Integration und Continuous Delivery.
    \item wie die Daten gespeichert und geladen werden können.
    \item wie ein benutzerfreundliches UI umgesetzt werden kann, damit User weitere Informationen zu Ausstellungsstücken bekommen.
\end{compactitem}

\subsection{Aufgabendifferenzierung}
\subsubsection{Ema}
\subsubsection{Lorenz}
\subsubsection{Fabian}
	
